\chapter{Background}

% In this chapter, all background necessary to understand the thesis are
% introduced. The level of detail is such that a colleague with similar background
% (no specialist!) is capable of understanding the contribution and impact of the
% thesis. A discussion of state-of-the-art solutions (e.g. literature research) is
% often helpful. Problems of the state-of-the-art are typically discussed and the
% contribution of the thesis is introduced in detail. 

In this chapter, we first present related work on scheduling in NCS scenarios in
Section \ref{sec:survey}. Afterwards, we give a short introduction to the
Gilbert-Elliot channel model in Section. \ref{sec:GE} and introduce the Dynamic
Programming algorithm for stochastic optimization in Section. \ref{sec:DP}.

\section{Related Work} \label{sec:survey}

Scheduling for NCSs has initially gained research interest from the control
community, where it has been studied as time-triggered and event-triggered
control problems \cite{molin2012optimality, molin2014price}. Both papers
consider optimizations for the steady-state of an NCS with respect to resource
constraints. While the proposed methods ensure optimal steady-state behavior,
the network is often assumed control-agnostic and is abstracted. However, 
the time-varying nature of wireless channels, the trade-offs among different
control-loops and the coexistence of heterogenous traffic types imply that
incorporating control metrics in network design can achieve performance benefits
on the whole system. \\
In communication research a trend towards cross-layer network design has been
shown to be beneficial in NCS settings \cite{park2017wireless}. Error reports
were the first cross-layer metric considered in a centralized scheduling problem
resources with multiple NCS sharing a communication channel
\cite{walsh2001scheduling}. Here, each sensor transmits its estimation error to
the scheduler. The scheduler then allocates free network resources to
sub-systems starting from the ones with maximum-error. In order to account for
underlying time-critical requirements in NCS applications,
\cite{vilgelm2017control} compares control-agnostic scheduling to control-aware
schedulers in a single-hop cellular NCS with varying channel qualities among
control loops. A heuristic scheduling policy is proposed which grants medium
access greedily to the sub-systems with the highest expected error. They have
shown, their control-aware scheduler outperforms an agnostic controller in terms
of maximum throughput with respect to QoC. \cite{vasconcelos2017optimal}
introduces \textit{one-shot} joint scheduling and estimation problem under
resource constraints. Here, a network is shared by multiple sensor and estimator
pairs. Given the probabilistic distributions of individual states, the
centralized scheduler selects a single sensor-estimator pair to transmit. The
authors have shown that it is globally optimal to choose the maximum quadratic
norm as scheduling and mean-value estimation as the estimation strategy. As the
name one-shot suggests, the paper focuses on a single transmission decision and
does not consider application-dependent propagation of estimation error over
multiple time-steps.

This changed in recent years with the introduction of AoI \cite{kaul2012real}.
Due to its ability to connect control and communication layers, AoI has
generalized joint design in NCS. In \cite{kadota2018scheduling}, a multi-user
scenario in which each user is prone to a time-invariant packet loss is
considered. They formulate a discrete-time wireless scheduling problem and show
analytically that the minimum AoI is achieved by updating the user with the
highest AoI. While AoI itself cannot be directly mapped into closed loop control
applications as AoI by design evolves linear in time and uniform for all
applications. Therefore, \cite{kosta2017age} proposed a non-linear aging concept
called \textit{value-of-information} (VoI) which take individual application
dynamics into account. \\
AoI based application-dependent age penalties, has first been adopted in
\cite{ayan2019age} for a centralized two-hop uplink and downlink scheduling
problem with heterogenous feedback control loops sharing a cellular network. A
non-linear age dynamic is derived by using AoI as an intermediate metric and
utilizing application specific system parameters. The scheduler accounts for
heterogenous NCS as the proposed age-penalty captures both the evolution of
individual system uncertainty and estimation error over time. The authors show
that optimal AoI results in worse control performance compared to their proposed
greedy VoI scheduling. They extend their findings in \cite{ayan2020optimal} for
multiple heterogeneous NCS but in a single hop scenario and propose an optimal
scheduler that minimizes the discounted age-penalty over an infinite time
horizon. However, similar to existing literature, this solution assumes packet
loss to be constant over time. To consider time-varying channel conditions,
\cite{ayan2020aoi} utilizes the same discounted age-penalty but solves the
scheduling problem using dynamic Programming and stochastic optimization.
Obtaining such a scheduling policy is computationally expensive and merely
optimal for a finite horizon.

\section{Gilbert-Elliot Model} \label{sec:GE}

In a wireless communication environment, channel errors are bursty, location
dependent, and mobility dependent. These are due to radio propagation
impairments such as shadowing and multi-path fading, as well as interference
from neighboring systems and users. In addition, one must take into account that
users of a wireless network, do not perceive the same channel quality at all
times, where channel quality is considered high when its bit error rate (BER) is
low. Thus, there is one wireless channel (link) between each pair of spatially
distributed nodes (users). 

\begin{figure}[h]
  \centering
  \begin{tikzpicture}[->,>=latex]
  % Create nodes
  \node[state] (G) {$G$};
  \node[state] (B) [right = 2cm of G] {$B$};

  % 
  \path (G) edge [loop left, left] node {$1-f$} (G);
  \path (B) edge [loop right, right] node {$1-r$} (B);
  \path (G) edge [bend left, above] node {$f$} (B);
  \path (B) edge [bend left, below] node {$r$} (G);
\end{tikzpicture}
 
  \caption{The Markov chain for the Gilbert-Elliot model}
  \label{fig:GE_FSM}
\end{figure}

In the literature, the time-varying quality of a bursty error wireless channel
is commonly modeled using the \textit{Gilbert–Elliot} model (GE)
\cite{gilbert1960capacity, elliott1963estimates}. It is a widely used stochastic
model for describing bit error processes in transmission channels, where errors
are correlated. There exits several parameterization of this model, but we will
be using the specific Markov chain shown in Fig.~(\ref{fig:GE_FSM}). This model
is a two-state homogenous Markov chain where each of the two states corresponds
to high or low channel quality and is called good state $G$ or bad state $B$,
respectively. Each of them may generate errors as independent events at a state
dependent error rate $p_G$ in the good and $p_B$ in the bad state. As shown in
\cite{hasslinger2008gilbert}, in order to apply the GE model in data loss
processes, we consider the packet reception process as a sequence of bits: 0
stands for a successful arrival of a packet whereas 1 denotes a lost or
corrupted packet.

Let $q(t)$ denote the state at time $t$, then the GE model is defined by the
transition matrix $\boldsymbol{T}$

\begin{equation}
  \boldsymbol{T} = 
  \begin{bmatrix}
    1-f & f \\
    r & 1-r \\
  \end{bmatrix}
\end{equation}
\begin{align}
  f &= \Pr[q(t) = B \mid q(t-1) = G] \qquad \textrm{failure rate} \\
  r &= \Pr[q(t) = G \mid q(t-1) = B] \qquad \textrm{recovery rate}
\end{align}

where $f\in(0,1)$ and $r\in(0,1)$ are the transition probabilities between
states, respectively. With these definitions, the stationary state probabilities
of the good state $\pi_G$ and the bad state $\pi_B$ exist and can be defined as
follows:

\begin{equation}
  \pi_G = \frac{r}{f+r} \quad \textrm{and} \quad \pi_B = \frac{f}{f+r}
\end{equation}

The stationary state probabilities can be interpreted as the average percentage
of time, in which the channel is in the good or the bad state. Thus, the
\textit{average error probability} is obtained as:

\begin{equation}
  p_E = \pi_G p_G + \pi_B p_B
  \label{eq:avgLoss}
\end{equation}

\subsection*{Average Coherence Time}
Another important characteristic defined by the GE model is the mean sojourn
time, i.e., the average duration that the wireless channel stays in a state. In
common NCS scenarios, packet transmissions occur in discrete time slots. Hence,
the channel can only change its state in these time slots and the amount of time
spent in a state is a geometrically distributed random variable $\tau_G$ and
$\tau_B$. Given the state transition probabilities, the mean sojourn times are:

\begin{equation}
  T_G = \E[\tau_G] = \frac{1}{f} \quad \textrm{and} \quad T_B = \E[\tau_B] = \frac{1}{r}
\end{equation}

Table~(\ref{tab:sojournTime}) summarizes sojourn times measurements performed
under different GE parameterization. The measurements resembles the expected
statistical properties of a geometric distribution and confirms our mapping of
state transition probabilities to their corresponding mean sojourn time.
Throughout the rest of the thesis, we will refer $T_G$ and $T_B$ as
\textit{average coherence times}.

\begin{table}[h]
  \begin{center}
  \begin{tabular}{|p{3.5cm}|c|c|c|c|}
  \hline 
  & \multicolumn{2}{|c|}{\textbf{Time in Good}} &
  \multicolumn{2}{|c|}{\textbf{Time in Bad}} \\
  \hline
  \textbf{Failure rate} $f$ / \textbf{Recovery rate} $r$ & \textbf{mean} & \textbf{std} & \textbf{mean}
  & \textbf{std}\\
  \hline \hline
  0.3 & 3.34 & 2.74 & 3.32 & 2.77 \\
  \hline 
  0.1 & 10.0 & 9.56 & 10.0 & 9.29 \\
  \hline 
  0.03 & 33.4 & 32.6 & 34.1 & 34.1 \\
  \hline 
  0.01 & 94.3 & 89.5 & 98.1 & 99.6 \\
  \hline 
  \end{tabular}
  \caption[Measurement of average coherence time]{Measurement of average
  coherence times for different transition probabilities. Note that the state
  transition probabilities are chosen symmetrically for this measurement, i.e.
  $f=r$}
  \label{tab:sojournTime}
  \end{center}
  \end{table}


\section{Dynamic Programming} \label{sec:DP}

Dynamic programming (DP) is both a mathematical optimization method and a
computer programming method. The method was developed by Richard Bellman in the
1950s and has found applications in numerous fields, from aerospace engineering
to economics. The fundamental idea behind DP is to break down a complicated
problem into simpler subproblems, obtaining the solution by iteratively
combining solutions of small subproblems in a bottom-up-approach. \\ In the
context of this thesis, DP is applied in solving a wireless resource scheduling
problem. We will first describe the principles of the DP algorithm with a
general decision problem under stochastic uncertainty. To this end, we formulate
a broadly applicable model of optimal control of a dynamic system over a finite
number of stages $H$ (a finite horizon). Detailed application of DP to solve our
scheduling problem will be given in chapter~(\ref{sec:problem}).

\subsection*{Basic Problem of Decision}
A general problem of decision has two principal features. An underlying
discrete-time dynamic system and an \textit{additive} cost function $J$ to be
minimized. The dynamic system expresses the evolution of the system's state,
under the influence of decisions made at discrete instances of time. The system
has the form of

\begin{equation*}
  x_{k+1} = f_k(x_k, u_k, w_k), \quad k = 0,1,\dots,H-1,
\end{equation*}

where

\begin{table}[h]
  \centering
  \begin{tabular}{rl}
    $k$ & indexes discrete time, \\
    $x_k \in \mathbb{S}_k$  & is the system's state and summarizes past information, \\
    $u_k \in \mathbb{C}_k$ & is the control or decision variable to be selected at time $k$, \\
    $w_k \in \mathbb{D}_k$ & is a random parameter, also called disturbance or noise, \\
    $H \in \mathbb{Z}^+$ & is the horizon or number of times control is applied, 
  \end{tabular}
\end{table}

and $f_k$ is a function that describes the system's dynamics. The control $u_k$
is constrained to take values in a given nonempty subset $\mathbb{U}_k(x_k)
\subset \mathbb{C}_k$, which depends on the current state $x_k$, i.e., $u_k \in
\mathbb{U}_k(x_k), \forall x_k \in \mathbb{S}_k$ and $k$. The random disturbance
$w_k$ is characterized by an independent probability distribution $P_k(\cdot\mid
x_k,u_k)$ that may depend explicity on $x_k$ and $u_k$.

The cost function is \textit{additive} over time in the sense that the cost
acquired at time $k$, denoted by $g_k(x_k, u_k, w_k)$, accumulates over time.
Meaning, the total cost is defined as:

\begin{equation*}
  g_N(x_H) + \sum\limits_{k=0}^{H-1}{g_k(x_k, u_k, w_k)},
\end{equation*}

where $g_N(x_H)$ is the terminal cost incurred at the end of the process.
However, since randomness is introduced by $w_k$, the cost is a random variable,
making this a stochastic optimization problem. Therefore, we form the
expectation of the total cost and minimize the \textit{expected cost}

\begin{equation*}
  \E\left[g_H(x_H) + \sum\limits_{k=0}^{H-1}{g_k(x_k, u_k, w_k)}\right],
\end{equation*}

where the expectation is taken over the random variables $w_k$ and $x_k$.

Consider the class of policies (also called control laws) that consists of a
sequence of functions 

\begin{equation}
  \pi = \left\{\mu_o,\dots,\mu_H-1\right\},
\end{equation}

where $\mu_k$ maps states $x_k$ into control inputs $u_k = \mu_k(x_k)$ and is
such that $\mu_k(x_k) \in \mathbb{U}_k, \forall x_k \in \mathbb{S}_k$. Such
policies will be called \textit{admissible}. Given an initial state $x_0$ and an
admissible policy $\pi = \left\{\mu_o,\dots,\mu_H-1\right\}$, the expected cost
of starting at $x_0$ is

\begin{equation}
  J_\pi(x_0) = \E_\pi\left[g_H(x_H) + \sum\limits_{k=0}^{H-1}{g_k(x_k, \mu_k(x_k), w_k)}\right]
\end{equation}

Since the system can solely be manipulated by means of control inputs or
decision variables $u_k$, the goal is to find a sequence of $u_k$, such that the
total expected cost is minimal. In other words, we are interested in an optimal
policy $\pi^*$, such that for $J_\pi(x_0)$ the following counts:

\begin{equation}
  J_\pi^*(x_0) = \min_{\pi\in\Pi}J_\pi(x_0) = J^*(x_0), \label{eq:HStageProblem}
\end{equation}

where $\Pi$ is the set of all admissible policies. Throughout the remaining
chapters, we will refer to the minimization problem in Eq.
(\ref{eq:HStageProblem}) as the \textit{H-stage problem} and drop the subscript
$\pi$ for brevity.

\subsection*{Principle of Optimality}
DP rests on the \textit{principle of optimality}, which intuitively states that
every optimal policy consists only of optimal sub policies. Applying this
principle to our above formulated problem, yields: 

\begin{center}
\fbox{\parbox{0.9\textwidth}{
  Let $\pi^* = \left\{\mu_0^*,\mu_1^*,\dots,\mu_{H-1}^*\right\}$ be an optimal 
  policy for the basic decision problem. Consider the subproblem whereby we are 
  at $x_i$ at time $i$ and wish to minimize the \textit{cost-to-go} from the 
  time $i$ to time $H$, i.e., 
  \begin{equation*}
    \E\left[g_H(x_H) + \sum\limits_{k=i}^{H-1}{g_k(x_k, u_k,w_k)}\right]. 
  \end{equation*}
  Then the policy $\left\{\mu_i^*,\mu_{i+1}^*,\dots,\mu_{H+1}^*\right\}$ is
  optimal for this subproblem.
  }
}
\end{center}

For an intuitive analogy, suppose that the fastest route from Munich to Mannheim
passes through Stuttgart. The principle of optimality translates to the obvious
fact that the Stuttgart to Mannheim portion of the route is also the fastest
route for a trip that starts from Stuttgart and ends in Mannheim.

The principle of optimality suggests that an optimal policy can be constructed
in a piece-by-peace fashion. First construct an optimal policy for the ``tail
subproblem'' involving the last stage, then extend the optimal policy to the
``tail subproblem'' involving the last two stages, and continuing iteratively
until an optimal policy for the entire problem is constructed. The DP algorithm
is based on this idea:

\subsection*{The DP Algorithm} \label{sec:DPalgorithm}

Adapted from \cite{bertsekas1995dynamic}. For any initial state $x_0$, the
optimal cost $J^*(x_0)$ of the basic decision problem is equal to $J_0(x_0)$,
given by the last step of the following algorithm: 


\begin{center}
  \fbox{
  \parbox{0.9\textwidth}{
    Iterate backwards from $k=N-1$ to 0 
    \begin{gather}
      J_H(x_H) = g_H(x_H), \\
      J_k(x_k) = \min_{u_k\in\mathbb{U}_k(x_k)} \E_{w_k}\left[g_k(x_k,u_k,w_k) + J_{k+1}(f_k(x_k,u_k,w_k)) \right], \label{eq:DPalgorithm}
    \end{gather}
  
    where the expectation is taken with respect to the probability distribution
    of $w_k$, dependent on $x_k$ and $u_k$. Further, if $u_k^* = \mu_k^*(x_k)$
    minimizes the right side of Eq.~(\ref{eq:DPalgorithm}) for each $x_k$ and
    $k$, the policy $\pi^* = \left\{\mu_0^*,\dots,\mu_{H-1}^*\right\}$ is
    optimal.}}
\end{center}

