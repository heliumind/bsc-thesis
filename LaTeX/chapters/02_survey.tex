\chapter{Background}

% In this chapter, all background necessary to understand the thesis are
% introduced. The level of detail is such that a colleague with similar background
% (no specialist!) is capable of understanding the contribution and impact of the
% thesis. A discussion of state-of-the-art solutions (e.g. literature research) is
% often helpful. Problems of the state-of-the-art are typically discussed and the
% contribution of the thesis is introduced in detail. 

In this chapter, we first present related work on scheduling in NCS scenarios in
Section \ref{sec:survey}. Afterwards, we give a short introduction to the
Gilbert-Elliot channel model in Section. \ref{sec:GE} and introduce the Dynamic
Programming algorithm for stochastic optimization in Section. \ref{sec:DP}.

\section{Related Work} \label{sec:survey}

Scheduling for NCSs has initially gained research interest from the control
community, where it has been studied as time-triggered and event-triggered
control problems \cite{molin2012optimality, molin2014price}. Both papers
consider optimizations for the steady-state of an NCS with respect to resource
constraints. While the proposed methods ensure optimal steady-state behavior,
the network is often assumed control-agnostic and is abstracted. However, 
the time-varying nature of wireless channels, the trade-offs among different
control-loops and the coexistence of heterogenous traffic types imply that
incorporating control metrics in network design can achieve performance benefits
on the whole system. \\
In communication research a trend towards cross-layer network design has been
shown to be beneficial in NCS settings \cite{park2017wireless}. Error reports
were the first cross-layer metric considered in a centralized scheduling problem
resources with multiple NCS sharing a communication channel
\cite{walsh2001scheduling}. Here, each sensor transmits its estimation error to
the scheduler. The scheduler then allocates free network resources to
sub-systems starting from the ones with maximum-error. In order to account for
underlying time-critical requirements in NCS applications,
\cite{vilgelm2017control} compares control-agnostic scheduling to control-aware
schedulers in a single-hop cellular NCS with varying channel qualities among
control loops. A heuristic scheduling policy is proposed which grants medium
access greedily to the sub-systems with the highest expected error. They have
shown, their control-aware scheduler outperforms an agnostic controller in terms
of maximum throughput with respect to QoC. \cite{vasconcelos2017optimal}
introduces \textit{one-shot} joint scheduling and estimation problem under
resource constraints. Here, a network is shared by multiple sensor and estimator
pairs. Given the probabilistic distributions of individual states, the
centralized scheduler selects a single sensor-estimator pair to transmit. The
authors have shown that it is globally optimal to choose the maximum quadratic
norm as scheduling and mean-value estimation as the estimation strategy. As the
name one-shot suggests, the paper focuses on a single transmission decision and
does not consider application-dependent propagation of estimation error over
multiple time-steps.

This changed in recent years with the introduction of AoI \cite{kaul2012real}.
Due to its ability to connect control and communication layers, AoI has
generalized joint design in NCS. In \cite{kadota2018scheduling}, a multi-user
scenario in which each user is prone to a time-invariant packet loss is
considered. They formulate a discrete-time wireless scheduling problem and show
analytically that the minimum AoI is achieved by updating the user with the
highest AoI. While AoI itself cannot be directly mapped into closed loop control
applications as AoI by design evolves linear in time and uniform for all
applications. Therefore, \cite{kosta2017age} proposed a non-linear aging concept
called \textit{value-of-information} (VoI) which take individual application
dynamics into account. \\
AoI based application-dependent age penalties, has first been adopted in
\cite{ayan2019age} for a centralized two-hop uplink and downlink scheduling
problem with heterogenous feedback control loops sharing a cellular network. A
non-linear age dynamic is derived by using AoI as an intermediate metric and
utilizing application specific system parameters. The scheduler accounts for
heterogenous NCS as the proposed age-penalty captures both the evolution of
individual system uncertainty and estimation error over time. The authors show
that optimal AoI results in worse control performance compared to their proposed
greedy VoI scheduling. They extend their findings in \cite{ayan2020optimal} for
multiple heterogeneous NCS but in a single hop scenario and propose an optimal
scheduler that minimizes the discounted age-penalty over an infinite time
horizon. However, similar to existing literature, this solution assumes packet
loss to be constant over time. To consider time-varying channel conditions,
\cite{ayan2020aoi} utilizes the same discounted age-penalty but solves the
scheduling problem using dynamic Programming and stochastic optimization.
Obtaining such a scheduling policy is computationally expensive and merely
optimal for a finite horizon.

\input{chapters/background}
