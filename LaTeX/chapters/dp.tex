\section{Dynamic Programming} \label{sec:DP}

Dynamic programming (DP) is both a mathematical optimization method and a
computer programming method. The method was developed by Richard Bellman in the
1950s and has found applications in numerous fields, from aerospace engineering
to economics. The fundamental idea behind DP is to break down a complicated
problem into simpler subproblems, obtaining the solution by iteratively
combining solutions of small subproblems in a bottom-up-approach. \\ In the
context of this thesis, DP is applied in solving a wireless resource scheduling
problem. We will first describe the principles of the DP algorithm with a
general decision problem under stochastic uncertainty. To this end, we formulate
a broadly applicable model of optimal control of a dynamic system over a finite
number of stages $H$ (a finite horizon). Detailed application of DP to solve our
scheduling problem will be given in chapter~(\ref{sec:problem}).

\subsection*{Basic Problem of Decision}
A general problem of decision has two principal features. An underlying
discrete-time dynamic system and an \textit{additive} cost function $J$ to be
minimized. The dynamic system expresses the evolution of the system's state,
under the influence of decisions made at discrete instances of time. The system
has the form of

\begin{equation*}
  x_{k+1} = f_k(x_k, u_k, w_k), \quad k = 0,1,\dots,H-1,
\end{equation*}

where

\begin{table}[h]
  \centering
  \begin{tabular}{rl}
    $k$ & indexes discrete time, \\
    $x_k \in \mathbb{S}_k$  & is the system's state and summarizes past information, \\
    $u_k \in \mathbb{C}_k$ & is the control or decision variable to be selected at time $k$, \\
    $w_k \in \mathbb{D}_k$ & is a random parameter, also called disturbance or noise, \\
    $H \in \mathbb{Z}^+$ & is the horizon or number of times control is applied, 
  \end{tabular}
\end{table}

and $f_k$ is a function that describes the system's dynamics. The control $u_k$
is constrained to take values in a given nonempty subset $\mathbb{U}_k(x_k)
\subset \mathbb{C}_k$, which depends on the current state $x_k$, i.e., $u_k \in
\mathbb{U}_k(x_k), \forall x_k \in \mathbb{S}_k$ and $k$. The random disturbance
$w_k$ is characterized by an independent probability distribution $P_k(\cdot\mid
x_k,u_k)$ that may depend explicity on $x_k$ and $u_k$.

The cost function is \textit{additive} over time in the sense that the cost
acquired at time $k$, denoted by $g_k(x_k, u_k, w_k)$, accumulates over time.
Meaning, the total cost is defined as:

\begin{equation*}
  g_N(x_H) + \sum\limits_{k=0}^{H-1}{g_k(x_k, u_k, w_k)},
\end{equation*}

where $g_N(x_H)$ is the terminal cost incurred at the end of the process.
However, since randomness is introduced by $w_k$, the cost is a random variable,
making this a stochastic optimization problem. Therefore, we form the
expectation of the total cost and minimize the \textit{expected cost}

\begin{equation*}
  \E\left[g_H(x_H) + \sum\limits_{k=0}^{H-1}{g_k(x_k, u_k, w_k)}\right],
\end{equation*}

where the expectation is taken over the random variables $w_k$ and $x_k$.

Consider the class of policies (also called control laws) that consists of a
sequence of functions 

\begin{equation}
  \pi = \left\{\mu_o,\dots,\mu_H-1\right\},
\end{equation}

where $\mu_k$ maps states $x_k$ into control inputs $u_k = \mu_k(x_k)$ and is
such that $\mu_k(x_k) \in \mathbb{U}_k, \forall x_k \in \mathbb{S}_k$. Such
policies will be called \textit{admissible}. Given an initial state $x_0$ and an
admissible policy $\pi = \left\{\mu_o,\dots,\mu_H-1\right\}$, the expected cost
of starting at $x_0$ is

\begin{equation}
  J_\pi(x_0) = \E_\pi\left[g_H(x_H) + \sum\limits_{k=0}^{H-1}{g_k(x_k, \mu_k(x_k), w_k)}\right]
\end{equation}

Since the system can solely be manipulated by means of control inputs or
decision variables $u_k$, the goal is to find a sequence of $u_k$, such that the
total expected cost is minimal. In other words, we are interested in an optimal
policy $\pi^*$, such that for $J_\pi(x_0)$ the following counts:

\begin{equation}
  J_\pi^*(x_0) = \min_{\pi\in\Pi}J_\pi(x_0) = J^*(x_0), \label{eq:HStageProblem}
\end{equation}

where $\Pi$ is the set of all admissible policies. Throughout the remaining
chapters, we will refer to the minimization problem in Eq.
(\ref{eq:HStageProblem}) as the \textit{H-stage problem} and drop the subscript
$\pi$ for brevity.

\subsection*{Principle of Optimality}
DP rests on the \textit{principle of optimality}, which intuitively states that
every optimal policy consists only of optimal sub policies. Applying this
principle to our above formulated problem, yields: 

\begin{center}
\fbox{\parbox{0.9\textwidth}{
  Let $\pi^* = \left\{\mu_0^*,\mu_1^*,\dots,\mu_{H-1}^*\right\}$ be an optimal 
  policy for the basic decision problem. Consider the subproblem whereby we are 
  at $x_i$ at time $i$ and wish to minimize the \textit{cost-to-go} from the 
  time $i$ to time $H$, i.e., 
  \begin{equation*}
    \E\left[g_H(x_H) + \sum\limits_{k=i}^{H-1}{g_k(x_k, u_k,w_k)}\right]. 
  \end{equation*}
  Then the policy $\left\{\mu_i^*,\mu_{i+1}^*,\dots,\mu_{H+1}^*\right\}$ is
  optimal for this subproblem.
  }
}
\end{center}

For an intuitive analogy, suppose that the fastest route from Munich to Mannheim
passes through Stuttgart. The principle of optimality translates to the obvious
fact that the Stuttgart to Mannheim portion of the route is also the fastest
route for a trip that starts from Stuttgart and ends in Mannheim.

The principle of optimality suggests that an optimal policy can be constructed
in a piece-by-peace fashion. First construct an optimal policy for the ``tail
subproblem'' involving the last stage, then extend the optimal policy to the
``tail subproblem'' involving the last two stages, and continuing iteratively
until an optimal policy for the entire problem is constructed. The DP algorithm
is based on this idea:

\subsection*{The DP Algorithm} \label{sec:DPalgorithm}

Adapted from \cite{bertsekas1995dynamic}. For any initial state $x_0$, the
optimal cost $J^*(x_0)$ of the basic decision problem is equal to $J_0(x_0)$,
given by the last step of the following algorithm: 


\begin{center}
  \fbox{
  \parbox{0.9\textwidth}{
    Iterate backwards from $k=N-1$ to 0 
    \begin{gather}
      J_H(x_H) = g_H(x_H), \\
      J_k(x_k) = \min_{u_k\in\mathbb{U}_k(x_k)} \E_{w_k}\left[g_k(x_k,u_k,w_k) + J_{k+1}(f_k(x_k,u_k,w_k)) \right], \label{eq:DPalgorithm}
    \end{gather}
  
    where the expectation is taken with respect to the probability distribution
    of $w_k$, dependent on $x_k$ and $u_k$. Further, if $u_k^* = \mu_k^*(x_k)$
    minimizes the right side of Eq.~(\ref{eq:DPalgorithm}) for each $x_k$ and
    $k$, the policy $\pi^* = \left\{\mu_0^*,\dots,\mu_{H-1}^*\right\}$ is
    optimal.}}
\end{center}
