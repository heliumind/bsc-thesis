\chapter{Introduction}

% This chapter should give a short overview over the whole thesis. It should
% provide background information on the thesis topic, introduce the task
% definition and give a short outlook on the rest of the thesis. 

The development of the upcoming generation of communication networks are largely
driven by changing application demands. Instead of solely focusing on data rate
increase, these networks are envisioned to support machine-type-communications
(MTC) or machine-to-machine communications (M2M), transforming the current
``Internet of Information'' to an ``Internet of Things and Services''. Prominent
applications include vehicular networks, industrial automation, tele-robotics,
smart grids and cyber-physical systems \cite{murray2003future}. From a system
theoretical view, most of these emerging applications fall into the category of
\textit{Networked Control Systems} (NCS), i.e., feedback control loops closed
over a communication network. Each loop consists of a plant, a sensor measuring
the plant’s output, and the respective controller reacting to the sensor’s
data.
\\
In a typical NCS scenario, multiple heterogenous NCSs share a wireless
communication medium and compete for network resources to transmit their latest
sensor measurements to the controllers. In order to manage limited network
resources and to avoid collisions, the access to the network is granted by a
scheduler that determines which subset of control loops are allowed to send
their up-to-date state information. In such a setting, the communication system
needs to satisfy different time-critical-requirements of the underlying control
loops, while dealing with problems inherent to the network, such as random
delays, packet losses and time varying channel conditions. These shortcomings
motivate \textit{control-aware} communication protocol design incorporating
prioritization and efficient scheduling of NCSs. Such schedulers aim to mitigate
these network induced challenges, which would otherwise result in a reduced
precision of control actions and degraded control quality
\cite{vasconcelos2017optimal}. Thus, networking policies need to be designed in
a joint fashion by using \textit{cross-layer} metrics and considering detailed
models of feedback control loops. 

Clearly, the control performance of NCS, i.e., quality of control (QoC), is
tightly coupled with the service provided by the communication network. While
performance of network systems are traditionally measured by means of delay,
jitter and throughput, these human-oriented metrics do not sufficiently capture
the QoC requirements of heterogeneous NCS applications. Hence, new cross-layer
metrics are needed in control-aware communication protocol design.
Age-of-Information (AoI) is such a relatively new metric that measures the
information freshness at the receiver monitoring a remote process
\cite{kaul2012real} and is used as a cross-layer metric in wireless medium
access protocol design. It is defined as the time elapsed since the generation
of the latest received information. As the name implies, AoI increases linearly
in time for all types of applications and drops upon receiving a new update. AoI
combines packet generation frequency, end-to-end delay, and packet loss in a
single metric. For instance, the absence of information increases AoI on the
controller side regardless of its cause: high delay, packet loss, or a low
information update frequency. As an interface between control application and
communication network, AoI has been widely adopted as an intermediate metric to
calculate control system metrics. 

% While such a setting allows control over large distances, wireless networks
% inevitably introduce random delays, packet losses and time varying channel
% conditions, degrading the control quality. As modern control theory is based on
% the assumption that information are transmitted along perfect communication
% channels, imperfections of the wireless channel or time-critical requirements of
% the underlying control loops impose challenges for both communication and
% control. 

\section{Related Work} \label{sec:survey}

% Scheduling for NCSs has initially gained research interest from the control
% community, where it has been studied as time-triggered and event-triggered
% control problems \cite{molin2012optimality, molin2014price}. Both papers
% consider optimizations for the steady-state of an NCS with respect to resource
% constraints. While the proposed methods ensure optimal steady-state behavior,
% the network is often assumed control-agnostic and is abstracted. However, 
% the time-varying nature of wireless channels, the trade-offs among different
% control-loops and the coexistence of heterogenous traffic types imply that
% incorporating control metrics in network design can achieve performance benefits
% on the whole system. \\
In communication research, a trend towards cross-layer network design has been
shown to be beneficial for NCS settings \cite{park2017wireless}. Error reports
were the first cross-layer metric considered in a centralized resource
scheduling problem with multiple NCS sharing a communication channel
\cite{walsh2001scheduling}. Here, each sensor transmits its estimation error to
the scheduler. The scheduler then allocates free network resources to
sub-systems starting from the ones with maximum-error. In order to account for
underlying time-critical requirements of NCS applications,
\cite{vilgelm2017control} compares control-agnostic scheduling to control-aware
schedulers in a single-hop cellular NCS with varying channel qualities among
control loops. A heuristic scheduling policy is proposed which grants medium
access greedily to the sub-systems with the highest expected error. They have
shown that a control-aware scheduler outperforms an agnostic controller in terms
of maximum throughput with respect to QoC. \cite{vasconcelos2017optimal}
introduces \textit{one-shot} joint scheduling and studies remote estimation
under resource constraints. Here, a network is shared by multiple sensor and
estimator pairs. Given the probabilistic distributions of individual states, the
centralized scheduler selects a single sensor-estimator pair to transmit. The
authors have shown that under constant link quality it is globally optimal to
choose the maximum quadratic norm as scheduling and mean-value estimate as the
estimation strategy. As the name one-shot suggests, the paper focuses on a
single transmission decision and does not consider application-dependent
propagation of estimation error over multiple time-steps.

This changed in recent years with the introduction of AoI \cite{kaul2012real}.
Due to its ability to connect application and communication layers, AoI has
generalized joint design in NCS. In \cite{kadota2018scheduling}, a multi-user
scenario in which each user is prone to a time-invariant packet loss is
considered. They formulate a discrete-time wireless scheduling problem and show
analytically that the minimum AoI is achieved by updating the user with the
highest AoI. Since AoI evolves linear in time and is uniform for all
applications by design, it cannot be directly mapped into closed loop control.
Therefore, \cite{kosta2017age} proposed a non-linear aging concept called
\textit{value-of-information} (VoI) which takes individual application dynamics
into account. \\
AoI-based application dependent age penalties have first been adopted in
\cite{ayan2019age} for a centralized two-hop uplink and downlink scheduling
problem with heterogenous feedback control loops sharing a cellular network. A
non-linear age dynamic is derived by using AoI as an intermediate metric and
utilizing application specific system parameters. The scheduler accounts for
heterogenous NCS as the proposed age-penalty captures both the propagation of
individual system uncertainties and estimation error over time. The authors show
that optimal AoI scheduling results in worse control performance compared to
their proposed greedy VoI scheduling. They extend their findings in
\cite{ayan2020optimal} for multiple heterogeneous NCS in a single hop scenario
and propose an optimal scheduler that minimizes a discounted age-penalty over an
infinite time horizon. The discount factor takes the importance of short-term
and long-term penalties into consideration, hence controlling the scheduler's
farsightedness. However, similar to existing literature, this solution assumes a
Bernoulli loss process. To consider time-varying channel conditions,
\cite{ayan2020aoi} assumes i.i.d. packet loss described by a rectified Gaussian
distribution. Although time-varying channel conditions prevent the calculation
of infinite-horizon scheduling policies, they provide an online scheduling
policy that is age-optimal for a finite horizon using dynamic programming and
stochastic optimization. Nonetheless, obtaining such a policy is computationally
expensive and merely optimal for a finite horizon.

\section*{Contributions}
In this work, we aim to develop an optimal scheduling policy addressing a
centralized wireless resource scheduling problem for NCS. We consider multiple
heterogenous feedback control loops sharing a wireless link with time-varying
channel conditions according to the Gilbert-Elliot Model. The deviation of the
real state from the estimated state on the receiver, i.e. controller, is taken
as performance metric. To provide optimal NCS behavior, we utilize AoI-based,
control dependent age-penalties and form scheduling decisions by means of
expected cost minimization. For an equivalent scenario, \cite{ayan2020aoi} has
proposed an online, centralized scheduling policy that is age-penalty optimal
for a finite horizon $H$. However, similar to most existing works, this solution
assumes a channel with independent and identically distributed (i.i.d) packet
loss. Wireless channels on the other hand are known to generate burst packet
losses/errors, meaning in reality packet losses tend to be correlated. One
simple model capturing correlated losses found in wireless fading channels is
the Gilbert-Elliot Model. After studying the impact of realistic wireless
channels on the state-of-the-art scheduler, we extend it to be fully
channel-aware. 

