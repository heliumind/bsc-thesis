\chapter{Introduction}

Development of the upcoming generation of communication networks are largely
driven by changing application demands. Instead of soley focusing on data rate
increase, these networks are envisioned to support machine-type-communications
(MTC) or machine-to-machine communications (M2M), transforming the current
``Internet of Information'' to a ``Internet of Things and Services''. Prominent
applications include vehicular networks, industrial automation, tele-robotics,
smart grids and cyber-physical systems. From a system theoretical view, most of
these emerging applications fall into the framework of \textit{Networked Control
Systems} (NCS),  i.e., feedback control loops closed over a communication
network. Each loop consists of a plant, a sensor, measuring the plant's output,
and the respective controller, reacting to the sensor's data. 

In a typical NCS scenario, multiple heterogenous NCSs share a wireless
communication medium and compete for network resource to transmit their latest
sensor measurements to the controllers. In such a scenario, the communication
system needs to satisfy different time-critical-requirements of the underlying
control loops, while dealing with network induced problems such as random
delays, packet losses and time varying channel conditions. To cope with these
challenges and not degrade control quality, \textit{cross-layer} networking
policies need to be adapted by considering detailed models of feedback control
loops. While performance of NCS, i.e., quality of control (QoC), is tightly
coupled with the service provided by the communication network, performance of
traditional network systems are extensively measured by means of delay, jitter
and throughput. However, these human-oriented metrics are not sufficient to
capture the QoC requirements of heterogeneous NCS applications. Thus, new
cross-layer metrics have been adopted in control-aware communication protocol
design.

Age-of-Information (AoI) is such a relatively new metric that measures the
information freshness at the receiver monitoring a remote process
\cite{kaul2012real} and used as cross-layer metric in wireless medium access
protocol design. It is defined as the time elapsed since the generation of
the latest received information. As the name suggests, AoI increases linearly in
time for all types of applications and drops upon receiving a new update. AoI
combines packet generation frequency, end-to-end delay, and packet loss in a
single metric. For instance the absence of information increases AoI on the
controller side regardless of its cause: high delay, packet loss, or a low
information update frequency. Due to its ability to connect control and
communication layers, AoI has been widely adopted as an intermediate metric to
calculate control system metrics. 

% While such a setting allows control over large distances, wireless networks
% inevitably introduce random delays, packet losses and time varying channel
% conditions, degrading the control quality. As modern control theory is based on
% the assumption that information are transmitted along perfect communication
% channels, imperfections of the wireless channel or time-critical requirements of
% the underlying control loops impose challenges for both communication and
% control. 

\section*{Problem Statement}
In this work, a centralized wireless resource scheduling problem for NCS with
time varying Gilbert-Elliot channel is adressed. To this end, we study control
dependent age-penalty minimization of feedback control loops that share a
wireless communication link with time-varying packet loss probabilities. Time
varying conditions in the network prevent the calculation of infinite horizon
optimal scheduling policies. Nevertheless, by applying stochastic optimization
and dynammic programming first works have proposed an online, centralized
scheduling policy that is age-penalty optimal for a finite horizon $H$. However,
this solution do not consider realistic channel behaviour as found in the
Gilbert-Elliot model. We aim to extend the state-of-the-art by combining our
findings of the Gilbert-Elliot channel model with the said AoI-based Finite
Horizon scheduler.

