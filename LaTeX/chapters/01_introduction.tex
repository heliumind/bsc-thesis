\chapter{Introduction}

% This chapter should give a short overview over the whole thesis. It should
% provide background information on the thesis topic, introduce the task
% definition and give a short outlook on the rest of the thesis. 

Development of the upcoming generation of communication networks are largely
driven by changing application demands. Instead of solely focusing on data rate
increase, these networks are envisioned to support machine-type-communications
(MTC) or machine-to-machine communications (M2M), transforming the current
``Internet of Information'' to a ``Internet of Things and Services''. Prominent
applications include vehicular networks, industrial automation, tele-robotics,
smart grids and cyber-physical systems \cite{murray2003future}. From a system
theoretical view, most of these emerging applications fall into the category of
\textit{Networked Control Systems} (NCS), i.e., feedback control loops closed
over a communication network. Each loop consists of a plant, a sensor, measuring
the plant's output, and the respective controller, reacting to the sensor's
data. \\
In a typical NCS scenario, multiple heterogenous NCSs share a wireless
communication medium and compete for network resources to transmit their latest
sensor measurements to the controllers. Medium access is granted by a
centralized scheduler that determines which subset of control loops are allowed
to send their up-to-date state information. In such a setting, the communication
system needs to satisfy different time-critical-requirements of the underlying
control loops, while dealing with problems inherent to the network, such as
random delays, packet losses and time varying channel conditions. These
shortcomings motivate \textit{control-aware} communication protocol design
incorporating prioritization and efficient scheduling of NCSs. Such schedulers
aim to mitigate these network induced challenges, which would otherwise result
in reduced precision of control actions and degraded control quality. Thus,
networking policies need to be designed in a joint fashion by using
\textit{cross-layer} metrics and considering detailed models of feedback control
loops. 

Clearly, the control performance of NCS, i.e., quality of control (QoC), is
tightly coupled with the service provided by the communication network. While
traditionally, performance of network systems are measured by means of delay,
jitter and throughput, these human-oriented metrics do not sufficiently capture
the QoC requirements of heterogeneous NCS applications. Hence, new cross-layer
metrics are needed in control-aware communication protocol design.
Age-of-Information (AoI) is such a relatively new metric that measures the
information freshness at the receiver monitoring a remote process
\cite{kaul2012real} and is used as a cross-layer metric in wireless medium
access protocol design. It is defined as the time elapsed since the generation
of the latest received information. As the name implies, AoI increases linearly
in time for all types of applications and drops upon receiving a new update. AoI
combines packet generation frequency, end-to-end delay, and packet loss in a
single metric. For instance, the absence of information increases AoI on the
controller side regardless of its cause: high delay, packet loss, or a low
information update frequency. As an interface between control application and
communication network, AoI has been widely adopted as an intermediate metric to
calculate control system metrics. 

% While such a setting allows control over large distances, wireless networks
% inevitably introduce random delays, packet losses and time varying channel
% conditions, degrading the control quality. As modern control theory is based on
% the assumption that information are transmitted along perfect communication
% channels, imperfections of the wireless channel or time-critical requirements of
% the underlying control loops impose challenges for both communication and
% control. 

\section*{Problem Statement}
In this work, we aim to develop an optimal scheduling policy addressing a
centralized wireless resource scheduling problem for NCS. We consider multiple
heterogenous feedback control loops sharing a wireless link with time-varying
channel conditions according to the Gilbert-Elliot Model. The deviation of the
real state from the estimated state on the receiver, i.e. controller, is taken
as performance metric. To provide optimal NCS behavior, we utilize AoI-based,
control dependent age-penalties and form scheduling decisions by means of
expected cost minimization. For a similar scenario, \cite{ayan2020aoi} has
proposed an online, centralized scheduling policy that is age-penalty optimal
for a finite horizon $H$. However, similar to most existing works, this solution
assumes a channel with independent and identically distributed (i.i.d) packet
loss. Wireless channels on the other hand are known to generate burst packet
losses/errors, meaning in reality packet losses tend to be correlated. One
simple model capturing correlated losses found in wireless fading channels is
the Gilbert-Elliot Model. We aim to extend the state-of-the-art by combining our
findings of the Gilbert-Elliot Channel Model with the said AoI-based Finite
Horizon scheduler.

