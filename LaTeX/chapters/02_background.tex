\chapter{Background}

% In this chapter, all background necessary to understand the thesis are
% introduced. The level of detail is such that a colleague with similar background
% (no specialist!) is capable of understanding the contribution and impact of the
% thesis. A discussion of state-of-the-art solutions (e.g. literature research) is
% often helpful. Problems of the state-of-the-art are typically discussed and the
% contribution of the thesis is introduced in detail. 

% In this chapter, we first present related work on scheduling in NCS scenarios in
% Section~\ref{sec:survey}. Afterwards, we give a short introduction to the
% Gilbert-Elliot Channel Model, its interpretation and statistical properties in
% Section.~\ref{sec:GE}. 

\section{Gilbert-Elliot Model} \label{sec:GE}

In a wireless communication environment, channel errors are bursty, location
dependent, and mobility dependent. These are due to radio propagation
impairments such as shadowing and multi-path fading, as well as interference
from neighboring systems and users. In addition, one must take into account that
users of a wireless network, do not perceive the same channel quality at all
times, where channel quality is considered high when its bit error rate (BER) is
low. Thus, there is one wireless channel (link) between each pair of spatially
distributed nodes (users). 

\begin{figure}[htb]
  \centering
  \begin{tikzpicture}[->,>=latex]
  % Create nodes
  \node[state] (G) {$G$};
  \node[state] (B) [right = 2cm of G] {$B$};

  % 
  \path (G) edge [loop left, left] node {$1-f$} (G);
  \path (B) edge [loop right, right] node {$1-r$} (B);
  \path (G) edge [bend left, above] node {$f$} (B);
  \path (B) edge [bend left, below] node {$r$} (G);
\end{tikzpicture}
 
  \caption{The Markov chain for the Gilbert-Elliot Model}
  \label{fig:GE_FSM}
\end{figure}

In the literature, the time-varying quality of wireless channel prone to bursty
errors is commonly modeled using the \textit{Gilbert–Elliot} Model (GE)
\cite{gilbert1960capacity, elliott1963estimates}. It is a widely used stochastic
model for describing correlated bit error processes in transmission channels.
There exits several parameterization of this model, but we will be using the
specific Markov chain shown in Fig.~\ref{fig:GE_FSM}. This model is a two-state
homogenous Markov chain where each of the two states corresponds to high or low
channel quality and is called good state $G$ or bad state $B$, respectively.
Each of them may generate errors as independent events at a state dependent
error rate $p_G$ in the good and $p_B$ in the bad state. In order to apply the
GE model in packet loss processes, we consider the packet reception process as a
sequence of bits: 0 stands for a successful arrival of a packet whereas 1
denotes a lost or corrupted packet \cite{hasslinger2008gilbert}.

Let $q(t)$ denote the state at time $t$, then the GE model is defined by the
stochastic transition matrix $\boldsymbol{T}$:

\begin{equation}
  \label{eq:GE_transition}
  \boldsymbol{T} = 
  \begin{bmatrix}
    1-f & f \\
    r & 1-r \\
  \end{bmatrix},
\end{equation}
\begin{align}
  f &= \Pr[q(t+1) = B \mid q(t) = G] \qquad \textrm{failure rate}, \\
  r &= \Pr[q(t+1) = G \mid q(t) = B] \qquad \textrm{recovery rate},
\end{align}

where $f\in(0,1)$ and $r\in(0,1)$ are the transition probabilities between good
and bad states, respectively. With these definitions, the stationary state
probabilities of the good state $\pi_G$ and the bad state $\pi_B$ exist and can
be defined as follows:

\begin{equation}
  \pi_G = \frac{r}{f+r} \quad \textrm{and} \quad \pi_B = \frac{f}{f+r}
\end{equation}

The stationary state probabilities can be interpreted as the average percentage
of time in which the channel is in $G$ or $B$. Thus, the \textit{average error
probability} is obtained as:

\begin{equation}
  p_E = \pi_G p_G + \pi_B p_B
  \label{eq:avgLoss}
\end{equation}

\subsection{Mean Sojourn Time}
Another important characteristic defined by the GE model is the mean sojourn
time, i.e., the average duration that the channel stays in a state. In common
NCS scenarios, packet transmissions occur in discrete time slots
\cite{park2017wireless}. We assume that the channel only changes its state
during these time slots. Then the amount of time spent in each state corresponds
to the number of independent Bernoulli experiments until a channel transition
occurs. Denoting sojourn times in good and bad with $\tau_G$ and $\tau_B$
respectively, the probability of staying $i$ slots in the good state, i.e.,
$\tau_G=i$, is $(1-f)^if$. Thus, $\tau_G$ and $\tau_B$ follow a geometrical
distribution and the mean sojourn times are its expectation: 

\begin{equation}
  T_G = \E[\tau_G] = \frac{1}{f} \quad \textrm{and} \quad T_B = \E[\tau_B] = \frac{1}{r}
\end{equation}

Table~(\ref{tab:sojournTime}) summarizes sojourn time measurements performed
under different GE parametrization. The measurement resembles the expected
statistical properties of a geometric distribution and confirms our mapping of
state transition probabilities to their corresponding mean sojourn time. Note
that both $f$ and $r$ control the ``burstiness'' of the modeled channel. For
instance, the smaller $r$ is, the longer the channel will stay in the bad state
on average. Hence, longer burst errors are to be expected. 

\begin{table}[htb]
  \begin{center}
  \begin{tabular}{|p{3.5cm}|c|>{\centering\arraybackslash}p{2.05cm}|c|>{\centering\arraybackslash}p{2.05cm}|}
  \hline 
  \textbf{Sojourn time in } & \multicolumn{2}{c|}{\textbf{Good}} &
  \multicolumn{2}{c|}{\textbf{Bad}} \\
  \hline
  \textbf{Failure rate} $f$ / \textbf{Recovery rate} $r$ & \textbf{Mean} &
  \textbf{Standard Deviation} & \textbf{Mean}
  & \textbf{Standard Deviation}\\
  \hline \hline
  0.3 & 3.34 & 2.74 & 3.32 & 2.77 \\
  0.1 & 10.0 & 9.56 & 10.0 & 9.29 \\
  0.03 & 33.4 & 32.6 & 34.1 & 34.1 \\
  0.01 & 94.3 & 89.5 & 98.1 & 99.6 \\
  \hline 
  \end{tabular}
  \caption[Measurement of mean sojourn time in Gilbert-Elliot
  channels]{Measurement of mean sojourn times for different transition
  probabilities. Note that the state transition probabilities are chosen
  symmetrically for this measurement, i.e. $f=r$}
  \label{tab:sojournTime}
  \end{center}
\end{table}
