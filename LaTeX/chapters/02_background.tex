\chapter{Background}

% In this chapter, all background necessary to understand the thesis are
% introduced. The level of detail is such that a colleague with similar background
% (no specialist!) is capable of understanding the contribution and impact of the
% thesis. A discussion of state-of-the-art solutions (e.g. literature research) is
% often helpful. Problems of the state-of-the-art are typically discussed and the
% contribution of the thesis is introduced in detail. 

In this chapter, we first present related work on scheduling in NCS scenarios in
Section~(\ref{sec:survey}). Afterwards, we give a short introduction to the
Gilbert-Elliot Channel Model in Section.~(\ref{sec:GE}). 

\section{Related Work} \label{sec:survey}

% Scheduling for NCSs has initially gained research interest from the control
% community, where it has been studied as time-triggered and event-triggered
% control problems \cite{molin2012optimality, molin2014price}. Both papers
% consider optimizations for the steady-state of an NCS with respect to resource
% constraints. While the proposed methods ensure optimal steady-state behavior,
% the network is often assumed control-agnostic and is abstracted. However, 
% the time-varying nature of wireless channels, the trade-offs among different
% control-loops and the coexistence of heterogenous traffic types imply that
% incorporating control metrics in network design can achieve performance benefits
% on the whole system. \\
In communication research a trend towards cross-layer network design has been
shown to be beneficial in NCS settings \cite{park2017wireless}. Error reports
were the first cross-layer metric considered in a centralized resource
scheduling problem with multiple NCS sharing a communication channel
\cite{walsh2001scheduling}. Here, each sensor transmits its estimation error to
the scheduler. The scheduler then allocates free network resources to
sub-systems starting from the ones with maximum-error. In order to account for
underlying time-critical requirements in NCS applications,
\cite{vilgelm2017control} compares control-agnostic scheduling to control-aware
schedulers in a single-hop cellular NCS with varying channel qualities among
control loops. A heuristic scheduling policy is proposed which grants medium
access greedily to the sub-systems with the highest expected error. They have
shown, their control-aware scheduler outperforms an agnostic controller in terms
of maximum throughput with respect to QoC. \cite{vasconcelos2017optimal}
introduces \textit{one-shot} joint scheduling and estimation problem under
resource constraints. Here, a network is shared by multiple sensor and estimator
pairs. Given the probabilistic distributions of individual states, the
centralized scheduler selects a single sensor-estimator pair to transmit. The
authors have shown that it is globally optimal to choose the maximum quadratic
norm as scheduling and mean-value estimation as the estimation strategy. As the
name one-shot suggests, the paper focuses on a single transmission decision and
does not consider application-dependent propagation of estimation error over
multiple time-steps.

This changed in recent years with the introduction of AoI \cite{kaul2012real}.
Due to its ability to connect control and communication layers, AoI has
generalized joint design in NCS. In \cite{kadota2018scheduling}, a multi-user
scenario in which each user is prone to a time-invariant packet loss is
considered. They formulate a discrete-time wireless scheduling problem and show
analytically that the minimum AoI is achieved by updating the user with the
highest AoI. While AoI itself cannot be directly mapped into closed loop control
applications due to AoI evolving linear in time and uniform for all applications
by design. Therefore, \cite{kosta2017age} proposed a non-linear aging concept
called \textit{value-of-information} (VoI) which takes individual application
dynamics into account. \\
AoI based application-dependent age penalties, has first been adopted in
\cite{ayan2019age} for a centralized two-hop uplink and downlink scheduling
problem with heterogenous feedback control loops sharing a cellular network. A
non-linear age dynamic is derived by using AoI as an intermediate metric and
utilizing application specific system parameters. The scheduler accounts for
heterogenous NCS as the proposed age-penalty captures both the propagation of
individual system uncertainties and of the estimation error over time. The
authors show that optimal AoI scheduling results in worse control performance
compared to their proposed greedy VoI scheduling. They extend their findings in
\cite{ayan2020optimal} for multiple heterogeneous NCS but in a single hop
scenario and propose an optimal scheduler that minimizes the discounted
age-penalty over an infinite time horizon. A discount factor takes the
importance of short-term and long-term penalties into consideration, hence,
controlling the scheduler's farsightedness. However, similar to existing
literature, this solution assumes a Bernoulli loss process. To consider
time-varying channel conditions, \cite{ayan2020aoi} assumes i.i.d. packet loss
described by a rectified Gaussian distribution. Although time-varying channel
conditions prevent the calculation of infinite-horizon scheduling policies, they
provide an online scheduling policy that is age-optimal for a finite horizon
using dynamic programming and stochastic optimization. Nonetheless, obtaining
such a policy is computationally expensive and merely optimal for a finite
horizon.

\section{Gilbert-Elliot Model} \label{sec:GE}

In a wireless communication environment, channel errors are bursty, location
dependent, and mobility dependent. These are due to radio propagation
impairments such as shadowing and multi-path fading, as well as interference
from neighboring systems and users. In addition, one must take into account that
users of a wireless network, do not perceive the same channel quality at all
times, where channel quality is considered high when its bit error rate (BER) is
low. Thus, there is one wireless channel (link) between each pair of spatially
distributed nodes (users). 

\begin{figure}[htb]
  \centering
  \begin{tikzpicture}[->,>=latex]
  % Create nodes
  \node[state] (G) {$G$};
  \node[state] (B) [right = 2cm of G] {$B$};

  % 
  \path (G) edge [loop left, left] node {$1-f$} (G);
  \path (B) edge [loop right, right] node {$1-r$} (B);
  \path (G) edge [bend left, above] node {$f$} (B);
  \path (B) edge [bend left, below] node {$r$} (G);
\end{tikzpicture}
 
  \caption{The Markov chain for the Gilbert-Elliot Model}
  \label{fig:GE_FSM}
\end{figure}

In the literature, the time-varying quality of a bursty error wireless channel
is commonly modeled using the \textit{Gilbert–Elliot} Model (GE)
\cite{gilbert1960capacity, elliott1963estimates}. It is a widely used stochastic
model for describing bit error processes in transmission channels, where errors
are correlated. There exits several parameterization of this model, but we will
be using the specific Markov chain shown in Fig.~(\ref{fig:GE_FSM}). This model
is a two-state homogenous Markov chain where each of the two states corresponds
to high or low channel quality and is called good state $G$ or bad state $B$,
respectively. Each of them may generate errors as independent events at a state
dependent error rate $p_G$ in the good and $p_B$ in the bad state. As shown in
\cite{hasslinger2008gilbert}, in order to apply the GE model in data loss
processes, we consider the packet reception process as a sequence of bits: 0
stands for a successful arrival of a packet whereas 1 denotes a lost or
corrupted packet.

Let $q(t)$ denote the state at time $t$, then the GE model is defined by the
transition matrix $\boldsymbol{T}$

\begin{equation}
  \label{eq:GE_transition}
  \boldsymbol{T} = 
  \begin{bmatrix}
    1-f & f \\
    r & 1-r \\
  \end{bmatrix},
\end{equation}
\begin{align}
  f &= \Pr[q(t+1) = B \mid q(t) = G] \qquad \textrm{failure rate}, \\
  r &= \Pr[q(t+1) = G \mid q(t) = B] \qquad \textrm{recovery rate},
\end{align}

where $f\in(0,1)$ and $r\in(0,1)$ are the transition probabilities between good
and bad states, respectively. With these definitions, the stationary state
probabilities of the good state $\pi_G$ and the bad state $\pi_B$ exist and can
be defined as follows:

\begin{equation}
  \pi_G = \frac{r}{f+r} \quad \textrm{and} \quad \pi_B = \frac{f}{f+r}
\end{equation}

The stationary state probabilities can be interpreted as the average percentage
of time, in which the channel is in the good or the bad state. Thus, the
\textit{average error probability} is obtained as:

\begin{equation}
  p_E = \pi_G p_G + \pi_B p_B
  \label{eq:avgLoss}
\end{equation}

\subsection{Average Coherence Time}
Another important characteristic defined by the GE model is the mean sojourn
time, i.e., the average duration that the wireless channel stays in a state. In
common NCS scenarios, packet transmissions occur in discrete time slots. Hence,
the channel can only change its state in these time slots and the amount of time
spent in a state is a geometrically distributed random variable $\tau_G$ and
$\tau_B$. Given the state transition probabilities, the mean sojourn times are:

\begin{equation}
  T_G = \E[\tau_G] = \frac{1}{f} \quad \textrm{and} \quad T_B = \E[\tau_B] = \frac{1}{r}
\end{equation}

Table~(\ref{tab:sojournTime}) summarizes sojourn times measurements performed
under different GE parameterization. The measurements resembles the expected
statistical properties of a geometric distribution and confirms our mapping of
state transition probabilities to their corresponding mean sojourn time. Note
that both $f$ and $r$ control the ``burstiness'' of the modeled channel. For
instance, the smaller $r$ is, the longer the channel will stay in the bad state
on average. Hence, longer burst errors are to be expected. Throughout the rest
of the thesis, we will refer $T_G$ and $T_B$ as \textit{average coherence
times}.

\begin{table}[h]
  \begin{center}
  \begin{tabular}{|p{3.5cm}|c|>{\centering\arraybackslash}p{2.05cm}|c|>{\centering\arraybackslash}p{2.05cm}|}
  \hline 
  & \multicolumn{2}{|c|}{\textbf{Time slots in Good}} &
  \multicolumn{2}{|c|}{\textbf{Time slots in Bad}} \\
  \hline
  \textbf{Failure rate} $f$ / \textbf{Recovery rate} $r$ & \textbf{Mean} &
  \textbf{Standard Deviation} & \textbf{Mean}
  & \textbf{Standard Deviation}\\
  \hline \hline
  0.3 & 3.34 & 2.74 & 3.32 & 2.77 \\
  \hline 
  0.1 & 10.0 & 9.56 & 10.0 & 9.29 \\
  \hline 
  0.03 & 33.4 & 32.6 & 34.1 & 34.1 \\
  \hline 
  0.01 & 94.3 & 89.5 & 98.1 & 99.6 \\
  \hline 
  \end{tabular}
  \caption[Measurement of average coherence time in Gilbert-Elliot
  channels]{Measurement of average coherence times for different transition
  probabilities. Note that the state transition probabilities are chosen
  symmetrically for this measurement, i.e. $f=r$}
  \label{tab:sojournTime}
\end{center}
\end{table}
