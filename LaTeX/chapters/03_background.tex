\section{Gilbert-Elliot Model}

In a wireless communication environment, channel errors are bursty, location
dependent, and mobility dependent. These are due to radio propagation
impairments such as shadowing and multipath fading, as well as interference from
neighboring systems and users. In addition, one must take into account that
users of a wireless network, do not perceive the same channel quality at all
times, where channel quality is considered high when its bit error rate (BER) is
low. Thus, there is one wireless channel (link) between each pair of spatially
distributed nodes (users). 

\begin{figure}[h]
  \centering
  \begin{tikzpicture}[->,>=latex]
  % Create nodes
  \node[state] (G) {$G$};
  \node[state] (B) [right = 2cm of G] {$B$};

  % 
  \path (G) edge [loop left, left] node {$1-f$} (G);
  \path (B) edge [loop right, right] node {$1-r$} (B);
  \path (G) edge [bend left, above] node {$f$} (B);
  \path (B) edge [bend left, below] node {$r$} (G);
\end{tikzpicture}
 
  \caption{The Markov chain for the Gilbert-Elliot model}
  \label{fig:GE_FSM}
\end{figure}

In the literature, the time-varying quality of a bursty error wireless channel
is commonly modeled using the \textit{Gilbert–Elliot} model (GE)
\cite{gilbert1960capacity}
\cite{elliott1963estimates}. It is a widely used stochastic model for describing
bit error processes in transmisson channels, where errors are correlated. There
are several parameterizations of this model, but we will use the specific Markov
chain shown in Fig. \ref{fig:GE_FSM}. This model is a two-state homogenous
Markov chain where each of the two states corresponds to high or low channel
quality and is called good state $G$ or bad state $B$, respectively. Each of them may
generate errors as independent events at a state dependent error rate $p_G$ in
the good and $p_B$ in the bad state. As shown in \cite{hasslinger2008gilbert},
to apply the model in data loss processes, we consider the packet reception
process as a sequence of bits: 0 stands for a successful arrival
of a packet whereas 1 denotes a lost or corrupted packet.

Let $q_t$ denote the state at time $t$, the GE model is defined by the
transition matrix $\Pi$

\begin{equation*}
  \Pi = \left (\begin{array}{cc} 1-f & f \\ r & 1-r \end{array} \right) 
\end{equation*}
\begin{align*}
  f &= P(q_t = B \mid q_{t-1} = G) \qquad \textrm{failure rate} \\
  r &= P(q_t = G \mid q_{t-1} = B) \qquad \textrm{recovery rate}
\end{align*}

where $f\in(0,1)$ and $r\in(0,1)$ are the transition probabilty between states,
respectively. With these definitions, the stationary state probabilties of the
good state $\pi_G$ and $\pi_B$ exist and can be defined as follows:

\begin{equation}
  \pi_G = \frac{r}{f+r} \quad \textrm{and} \quad \pi_B = \frac{f}{f+r}
\end{equation}

The stationary state probabilties can be interpreted as the average percentage
of time, in which the channel is in good or bad state. Thus, the \textit{average
error probabilty} is obtained as:

\begin{equation}
  p_E = \pi_G p_G + \pi_B p_B
  \label{eq:avgLoss}
\end{equation}

Another important characteristic defined by the GE model is the mean sojourn
time, i.e., the average duration that the wirless channel stays in a state. In
common NCS scenarios, packet transmissions occur in discrete time slots. Hence,
the channel can only change its state in these time slots and the amount of time
spent in a state is a geometrically distributed random variable $\tau_G$ and
$\tau_B$. Given the state transition probabilties, the mean sojourn times are:

\begin{equation}
  T_G = \E[\tau_G] = \frac{1}{f} \quad \textrm{and} \quad T_B = \E[\tau_B] = \frac{1}{r}
\end{equation}

Table \ref{tab:sojournTime} summarizes sojourn times measurements performed
under different GE parameterizations. The measurements resembles the expected
statistical properties of a geometric distribution and confirms our mapping of
state transition probabilties to their corresponding mean sojourn time, which we
will term \textit{average coherence time} throughout the rest of the thesis.

\begin{table}[h]
  \begin{center}
  \begin{tabular}{|p{3.5cm}|c|c|c|c|}
  \hline 
  & \multicolumn{2}{|c|}{\textbf{Time in Good}} &
  \multicolumn{2}{|c|}{\textbf{Time in Bad}} \\
  \hline
  \textbf{Failure rate} $f$ / \textbf{Recovery rate} $r$ & \textbf{mean} & \textbf{std} & \textbf{mean}
  & \textbf{std}\\
  \hline \hline
  0.3 & 3.34 & 2.74 & 3.32 & 2.77 \\
  \hline 
  0.1 & 10.0 & 9.56 & 10.0 & 9.29 \\
  \hline 
  0.03 & 33.4 & 32.6 & 34.1 & 34.1 \\
  \hline 
  0.01 & 94.3 & 89.5 & 98.1 & 99.6 \\
  \hline 
  \end{tabular}
  \caption{Measurment of sojourn times for different transition probabilties. Note that the state transition
  probabilties are chosen symmetrically for this measurement, i.e. $f=r$}
  \label{tab:sojournTime}
  \end{center}
  \end{table}


\section{Dynamic Programming Algorithm}

Dynamic programming (DP) is both a mathematical optimization method and a
computer programming method. The method was developed by Richard Bellman in the
1950s and has found applications in numerous fields, from aerospace engineering
to economics. In the context of this thesis, DP is applied in solving a wireless
resource scheduling problem. We will first describe the principles of the DP
algorithm with a general decision problem. To this end, we formulate a broadly
applicable model of optimal control of a dynamic system over a finite number of
stages (a finite horizon). Detailed application of DP to our scheduling problem
will be given in chapter. 

As such, it resembles a general problem of decision under stochastic
uncertainty, which has two principal features. An underlying discrete-time
dynamic system and a \textit{cost function}. Solving the decision problem
entails in minimizing this cost function, subject to the system's dynamics.
Given a discrete-time dynamic system of the form

\begin{equation*}
  x_{k+1} = f_k(x_k, u_k, w_k), \quad k = 0,1,\dots,N-1,
\end{equation*}

where

\begin{table}[h]
  \centering
  \begin{tabular}{rl}
    $k \in$ & indexes discrete time, \\
    $x_k \in$  & is the system's state and summarizes past information, \\
    $u_k \in$ & is the control or decision variable to be selected at time $k$, \\
    $w_k \in$ & is a random paramter, also called distrubance or noise, \\
    $N$ & is the horizon or number of times control is applied, 
  \end{tabular}
\end{table}

and $f_k$ is a function that describes the system's dynamics. The dynamic system expresses the evolution of the system's state, under the
influence of decisions made at discrete instances of time. 

The cost function is additive over time in the sense that the cost incurred at
time $k$, denoted by $g_k(x_k, u_k, w_k)$, accumulates over time. The total cost
is

\begin{equation*}
  g_N(x_N) + \sum\limits_{k=0}^{N-1}{g_k(x_k, u_k, w_k)},
\end{equation*}

where $g_N(x_N)$ is the terminal cost incurred at the end of the process.
However, since randomness introduced is by $w_k$, the cost is generally a
random variable and cannot be easily optimized. Therefore the optimal control
problem is formulated as an minimization of the \textit{expected cost}

\begin{equation*}
  J
\end{equation*}
