\chapter{Conclusion}
% The thesis is concluded here. The considered problem is repeated. The
% contribution of this work is highlighted and the results are recapitulated.
% Remaining questions are stated and ideas for future work are expressed. 

In this work, we have studied a centralized resource scheduling problem where
multiple NCSs share a wireless communication channel with time-varying packet
loss according to the Gilbert-Elliot channel model. Sub-systems compete for
network resources to improve the accuracy of their remote estimation process.
AoI was utilized as an interface between control applications and the network to
define characteristic age-penalty functions. Combining these insights, we have
proposed an AoI-based control-aware scheduling algorithm that takes the GE
channel fully into account. By looking $H$ slots into the future, the scheduler
builds a tree with every possible outcome. It then grants medium access to the
sub-system which is expected to incur the least age cost. We evaluated the
proposed method in simulations with time-varying and correlated packet loss
probabilities. In general, sub-systems produce more accurate estimations with
increased $H$ since the scheduler becomes more farsighted. On the other hand,
solving the $H$-stage problem with dynamic programming to obtain a global
optimal policy incurs exponential complexity. At the same time, we observed that
the performance gain achieved by increasing $H$ diminishes after $H=5$. Such a
point is obviously strongly scenario dependent but can be easily be found
heuristically. We conclude that the scheduler can be designed with a $H$ chosen
to mitigate complexity while still being able to provide adequate remote state
estimation performance. 

In the end, the simulation study has shown that making the state-of-the-art
finite horizon scheduler fully GE channel-aware by considering every possible
outcome results in exploding tree size and is thus not scalable. More effort is
needed to make the finite horizon scheduling scheme feasible in real-life
network conditions. While our proposed algorithm finds the global optimal
scheduling policy, it may be possible to find a trade-off between optimality and
complexity, as long as possible GE channel transitions are not neglected. A
compromise may be found by combining FHS and GES. Namely, we take the tree
structure of FHS but explore possible GE channel transitions by adapting the GES
branch transition probabilities on the certain levels. This way we still are GE
channel-aware without increasing complexity.

For simplicity, we assumed scalar sub-systems in our simulation setup. Further
research can be put into evaluating the efficiency of finite horizon scheduling
in real-life use-cases. An example would be a vehicular scenario case-study
where the users are applications from the vehicular domain, e.g. cruise control,
motor speed etc. 

% TODO: Add OS effect conclusion
