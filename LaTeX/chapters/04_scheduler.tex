\chapter{Scheduler Design}

In this chapter, we develop an online scheduling scheme that solves the $H$
stage-problem by employing Dynamic Programming and Stochastic Optimization.
First, Sec. \ref{sec:fhscheduler} summarizes the state-of-the-art Finite Horizon
Scheduler which our proposed scheduler is based on. We extend it to be
applicable in NCS scenarios with Gilbert-Elliot channels and give a short
complexity analysis in Sec. \ref{sec:gescheduler}.

\section{Finite Horizon Scheduler} \label{sec:fhscheduler}

In \cite{ayan2020aoi} a \textit{finite horizon scheduler (FHS)} is proposed
which takes optimal scheduling decisions for the next $H$ transmission slots. In
contrast to our scenario, independent losses drawn from a normal distribution
are considered, not accounting for GE channel state transitions, i.e.,
$\boldsymbol{s}(t) = \left[\boldsymbol{t}_g(t) \quad \boldsymbol{t}_r(t) \quad
\boldsymbol{t}_u(t) \right]^T$. This also means, that network state transitions
are solely dependent on packet loss probabilities.  

FHS solves the formulated $H$-stage problem by applying the Dynamic Programming
Algorithm (DP) \cite{bertsekas1995dynamic}. The fundamental idea behind DP is to
break down a complicated problem into simpler subproblems, obtaining the
solution by iteratively combining solutions of small subproblems in a
bottom-up-approach. This is motivated by the \textit{principle of optimality},
which intuitively states that every optimal policy consists only of optimal sub
policies. That is, assume $\pi^* = \{\mu^*(t), \mu^*(t+1), \dots, \mu^*(t+H-1)
\}$ to be an optimal policy for the $H$-stage problem. For the subproblem
whereby we are at at time $i>t$ and wish to minimize the \textit{cost-to-go}
from $\boldsymbol{s}(i)$ at time $i$ to time $H$, i.e.,

\begin{equation}
  \label{eq:costtogo}
  \min_{\pi\in\Pi}{J(\boldsymbol{s}(i)) = \E \left[ \sum_{t' = i }^{i+H} C(\boldsymbol{s}(t')) \right]},
\end{equation}

the policy $\left\{ \mu^*(i), \mu^*(i+1), \dots, \mu^*(t+H-1) \right\}$ is
optimal for this subproblem. Thus, an optimal scheduling policy can be
constructed in a piece-by-piece fashion by first solving the ``tail'' $H$-stage
subproblem involving stage $H-1$ and continuing iteratively backwards to stage
$0$. 

The FHS searches for an optimal policy in a tree structure, where each node
represents a network state $\boldsymbol{s}(t')$ at time $t'$ with $t \leq t'
\leq t+H$. The root of the tree corresponds to the current network state
$\boldsymbol{s}(t)$. In addition, nodes occurring in the same slot form a level,
where the root node is the 0-th level of the tree. Similarly, nodes in the last
level $H$ are called \textit{leaf nodes}. Each node is assigned with the
cost-to-go according to Eq.~\eqref{eq:costtogo} with its network state as
initial state $\boldsymbol{s}(i)$. By doing this, minimizing the total expected
cost means finding the shortest path in the described tree. The optimal
scheduling policy is then constructed from the scheduling decisions taken at
each level needed to take the shortest path.

Fig.~\ref{fig:FHStree} depicts an example tree with $H=1$ and $N=2$. Notice,
that for this example the optimal scheduling policy $\pi^*$ consists of only one
scheduling decision $\mu^*(t) \in \mathcal{M}(\boldsymbol{s}(t)) =
\left\{\varnothing, 1, 2 \right\}$ which determines the shortest path from Level
1 to 0 and achieves the optimal cost $J^*(\boldsymbol{s}(t))$ w.r.t.
Eq.~\eqref{eq:horizoncost}:

\begin{align*}
    \mu^*(t) =& \argmin_{\mu(t)\in\mathcal{M}(\boldsymbol{s}(t))}{J(\boldsymbol{s}(t))}
    = \argmin_{\mu(t)\in\mathcal{M}(\boldsymbol{s}(t))}{\E \left[ \sum_{t' = t }^{t + 1} C(\boldsymbol{s}(t')) \right]} \\
    =& \argmin_{\mu(t)\in\mathcal{M}(\boldsymbol{s}(t))}{} \left\{
      \underbrace{\Pr[0|\varnothing] C(\boldsymbol{s}_3)}_{\mu(t)=\varnothing}, \,
      \underbrace{\Pr[1|1] C(\boldsymbol{s}_1) + \Pr[0|1] C(\boldsymbol{s}_3)}_{\mu(t)=1}, \,
      \underbrace{\Pr[1|2] C(\boldsymbol{s}_2) + \Pr[0|2] C(\boldsymbol{s}_3)}_{\mu(t)=2} \right\}
\end{align*}

% \begin{equation*}
%   \begin{split}
%     \mu^*(t) = \argmin_{\mu(t)}{J(\boldsymbol{s}(t))}
%     = \argmin_{\mu(t)}{\E \left[ \sum_{t' = t }^{t + 1} C(\boldsymbol{s}(t')) \right]}
%     = \argmin_{\mu(t)}{}\left\{
%       \underbrace{\Pr[0|\varnothing] \cdot C(\boldsymbol{s}_3)}_{\mu(t)=\varnothing},\right. \\
%       \left. \underbrace{\Pr[1|1] \cdot C(\boldsymbol{s}_1) + \Pr[0|1] \cdot C(\boldsymbol{s}_3)}_{\mu(t)=1}, \,
%       \underbrace{\Pr[1|2] \cdot C(\boldsymbol{s}_2) + \Pr[0|2] \cdot C(\boldsymbol{s}_3)}_{\mu(t)=2} \right\}
%   \end{split}
% \end{equation*}

\begin{figure}[htb]
	\centering
  \resizebox{.8\columnwidth}{!}{\begin{tikzpicture}[>=latex]
\def\nodesize{1cm}

%Root
\node (s0) at (0,0) [draw, circle, align=center, inner sep=0pt, minimum size=\nodesize] {\scriptsize$\boldsymbol{s}(t)$};

%Level 1
\node (s1) at (-3,-2) [draw, circle, align=center, inner sep=0pt, minimum size=\nodesize] {\scriptsize$\boldsymbol{s}_1(t+1)$};

\node (s2) at (0,-2) [draw, circle, align=center, inner sep=0pt, minimum size=1cm] {\scriptsize$\boldsymbol{s}_2(t+1)$};
\node (s3) at (3,-2) [draw, circle, align=center, inner sep=0pt, minimum size=1cm] {\scriptsize$\boldsymbol{s}_3(t+1)$};

\draw[->] (s0.south west) [] to node [above, pos=0.5, xshift=-.9cm] (aa) {\tiny$\text{Pr}{[\gamma_{1}(t)=1|\mu(t)=1]}$} (s1.north east);
\draw[->] (s0.south) [] to node [pos=0.5, left] (bb) {\tiny$\text{Pr}{[1|2]}$} (s2.north);
\draw[->, bend right] (s0.south east) [] to node [pos=0.5, below, xshift=-.1cm] (bb) {\tiny$\text{Pr}{[0|1]}$} (s3.north west);
\draw[->, bend left] (s0.south east) [] to node [pos=0.5, above, yshift=0cm] (cc) {\tiny$\text{Pr}{[0|2]}$} (s3.north west);
\draw[->, bend left=60] (s0.east) [] to node [pos=0.5, above, yshift=0.1cm] (cc) {\tiny$\text{Pr}{[0|\varnothing] }$} (s3.north);

\node (lev0) [] at (-5,0) {\small Level 0};
\draw [dashed,mygray, very thick] (lev0.east) to node [] (xyz) {} (-.8,0);
\draw [dashed,mygray, very thick] (2.7,0) to node [] (xyz) {} (4.2,0);

\node (lev1) [] at (-5,-2) {\small Level 1};

\draw [dashed, mygray,very thick] (lev1.east) to node [] (xyz) {} (-3.7,-2);
\draw [dashed,mygray,very thick] (-2.2,-2) to node [] (xyz) {} (-.8,-2);
\draw [dashed,mygray,very thick] (.8,-2) to node [] (xyz) {} (2.2,-2);
\draw [dashed,mygray,very thick] (3.6,-2) to node [] (xyz) {} (4.2,-2);


%%\draw[->] (sM.south) [out=-140,in=-40] to node [midway, below] (m2) {$p_1$} (s0.south);
%
%\draw[->, dashed] (s1.north east) -- (4.2,3);
%\draw[->, dashed] (s1.east) [] to node [] (bbb) {} (4.2,2.3);
%\draw[->, dashed] (s1.east) [] to node [pos=0.4, above] (bbbb) {} (4.2,1.7);
%\draw[->, dashed] (s1.south east) -- (4.2,1.2);
%
%\draw[->, dashed] (s2.north east) -- (4.2,0.6);
%\draw[->, dashed] (s2.east) [] to node [pos=0.4, above] (bbb) {} (4.2,0.2);
%\draw[->, dashed] (s2.east) [] to node [pos=0.4, above] (bbb) {} (4.2,-0.2);
%%\draw[->, bend right, dashed] (s2.east) [] to node [pos=0.4, above] (bbbb) {} (4.3,1.7);
%%\draw[->, dashed] (s2.south east) -- (4.2,1.2);
\end{tikzpicture}
} 
  \caption[FHS: Example tree structure with 2 sub-systems and finite horizon
  1]{Example $1$ level tree structure with $2$ sub-systems, i.e., $H=1$ and
  $N=2$. Each edge is labeled with the corresponding transition probability,
  i.e., $\Pr{[\gamma_{\mu(t)}(t)\mid \mu(t)]}$ as the conditional probability
  for the possible next network state to occur given the scheduling decision.
  The states $\textbf{s}_1$ and $\textbf{s}_2$ represent the cases where
  scheduled sub-systems successfully transmit their latest observation to the
  controller while $\textbf{s}_3$ represents the failure case where every
  sub-system fails to receive the transmitted packet.}
	\label{fig:FHStree}
\end{figure}

\section{Channel-aware Finite Horizon Scheduler} \label{sec:gescheduler}

We extend the introduced FHS with a \textit{Gilbert-Elliot channel-aware
Scheduler (GES)}, that takes channel state transitions for each link into
consideration. Similarly, GES utilizes the DP algorithm in order to solve the
$H$-stage problem. Namely starting at state $\boldsymbol{s}(t)$ and time $t$,
the optimal final horizon cost $J^*(\boldsymbol{s}(t))$ is given by the last
step of the following algorithm:


\begin{center}
  \fbox{
    \parbox{0.9\textwidth}{
      Iterate backwards from $t'=t+H-1$ to $t'=t$
      \begin{equation} 
        \label{eq:dynamicprogramming}
        J_{t'}(\boldsymbol{s}(t')) = \min_{\mu(t') \in \mathcal{M}(\boldsymbol{s}(t'))} \E \left[ C(\boldsymbol{s}(t')) + J_{t' + 1}(\boldsymbol{s}(t' + 1))\right],
      \end{equation} 
      with the terminal cost: 
      \begin{equation}
        \label{eq:terminalcost}
        J_{t + H}(\boldsymbol{s}(t + H)) = C(\boldsymbol{s}(t + H)) 
      \end{equation}
      The expectation is taken with respect to the total MSE in the network as
      defined in Eq.~\eqref{eq:estimationerror} and Eq.~\eqref{eq:gfunction}.}
  }
\end{center}
  

According to the principle of optimality, if at each stage $t'$ the optimal
$\mu^*(t')$ action w.r.t to Eq.~\eqref{eq:dynamicprogramming} is taken, the
policy $\pi^* = \{\mu^*(t), \mu^*(t+1), \dots, \mu^*(t + H-1)\}$ is optimal and
achieves the minimum cost $J_{\pi^*}(\boldsymbol{s}(t)) = J_t(\boldsymbol{s}(t))
$. The proof can be found in \cite{bertsekas1995dynamic}.

\subsection{Algorithm}
The main difference to FHS lies in the size of the tree. The evolution of
network state $\boldsymbol{s}(t)$ with a GE channel not only depends on the
outcome of packet transmissions but is also simultaneously affected by the fact
that each link can change or retain their GE channel state $\boldsymbol{q}(t) $.
For instance, given $\delta_i(t)=1$, a successful transmission, i.e.,
$\gamma_i(t)=1$, can occur while Link $i$ changes or stays in Good/Bad state at
the same time. Provided any $\boldsymbol{s}(t')$, there exists $N$ individual
communication links. Thus, $2^N$ different GE channel transitions are possible
in the following time slot $t'+1$. Further, for each of these $2^N$ channel
states $\boldsymbol{q}(t'+1)$, $N+1$ network outcomes due to packet transmission
are possible, which correspond to $N$ successful transmissions for each link and
one common failed state. Compared to the FHS, each node in such a tree has $2^N$
times more child nodes. Fig.~(\ref{fig:GEStree}) shows a minimal tree for $N=2$,
$H=1$ and initial state $\boldsymbol{s}(t)=\left[t\quad t\quad a\quad b\quad
c\quad d\quad G\quad G \right]^T$.

In analogy to FHS, the backwards iteration employed by the DP algorithm can be
seen as traversing through all nodes of the tree level by level starting from
the leaf level while taking the optimal action at each level minimizing the
expected cost in the next level. Again, each node is assigned with a cost
obtained from Eq.~\eqref{eq:dynamicprogramming}. As a result, the operation of
the GE channel-aware FH scheduler can be summarized in the following steps:

\begin{enumerate}
	\item Initialize the current state $\boldsymbol{s}(t)$ as the root of a tree
	structure.
	\item Starting from the root, determine the possible actions at each node,
	i.e., $\mathcal{M}\boldsymbol{s}(t')$ for $\{t': t \leq t' < t + H \}$ and
	subsequently all possible next states $\boldsymbol{s}(t' + 1)$ when action
	$\mu(t')$ is taken.
	\item Add all possible next states as child nodes to the next level of the
	tree with the corresponding transition probabilities from the parent node.
	\item Repeat steps (2)-(4) until the $H$-th level of the tree is constructed.
	\item Assign costs to all states starting from the leaf nodes as in
	Eq.~\eqref{eq:dynamicprogramming} and \eqref{eq:terminalcost}.
\end{enumerate}

\begin{sidewaysfigure}
	\centering
  \begin{tikzpicture}[>=latex] 
\def\nodesize{1cm}
\def\leveliSize{-4cm}
\def\levelOneSize{-10cm}

%Root
\node (s0) at (0,0) [draw, circle, align=center, inner sep=0pt, minimum size=\nodesize] {\scriptsize$\boldsymbol{s}(t)$};

%Intermediate level
\node [dashed] (q1) at (-7.5,\leveliSize) [draw, circle, align=center, inner sep=0pt, minimum size=\nodesize] {\scriptsize$\boldsymbol{q}_1(t+1)$};
\node [dashed] (q2) at (-2.5,\leveliSize) [draw, circle, align=center, inner sep=0pt, minimum size=\nodesize] {\scriptsize$\boldsymbol{q}_2(t+1)$};
\node [dashed] (q3) at (2.5,\leveliSize) [draw, circle, align=center, inner sep=0pt, minimum size=\nodesize] {\scriptsize$\boldsymbol{q}_3(t+1)$};
\node [dashed] (q4) at (7.5,\leveliSize) [draw, circle, align=center, inner sep=0pt, minimum size=\nodesize] {\scriptsize$\boldsymbol{q}_4(t+1)$};

%Level 1
\node (s1) at (-9,\levelOneSize) [draw, circle, align=center, inner sep=0pt, minimum size=\nodesize] {\scriptsize$\boldsymbol{s}_1(t+1)$};
\node (s2) at (-7.5,\levelOneSize) [draw, circle, align=center, inner sep=0pt, minimum size=\nodesize] {\scriptsize$\boldsymbol{s}_2(t+1)$};
\node (s3) at (-6,\levelOneSize) [draw, circle, align=center, inner sep=0pt, minimum size=1cm] {\scriptsize$\boldsymbol{s}_3(t+1)$};
\node (s4) at (-4,\levelOneSize) [draw, circle, align=center, inner sep=0pt, minimum size=1cm] {\scriptsize$\boldsymbol{s}_4(t+1)$};
\node (s5) at (-2.5,\levelOneSize) [draw, circle, align=center, inner sep=0pt, minimum size=1cm] {\scriptsize$\boldsymbol{s}_5(t+1)$};
\node (s6) at (-1,\levelOneSize) [draw, circle, align=center, inner sep=0pt, minimum size=1cm] {\scriptsize$\boldsymbol{s}_6(t+1)$};
\node (s7) at (1,\levelOneSize) [draw, circle, align=center, inner sep=0pt, minimum size=1cm] {\scriptsize$\boldsymbol{s}_7(t+1)$};
\node (s8) at (2.5,\levelOneSize) [draw, circle, align=center, inner sep=0pt, minimum size=1cm] {\scriptsize$\boldsymbol{s}_8(t+1)$};
\node (s9) at (4,\levelOneSize) [draw, circle, align=center, inner sep=0pt, minimum size=1cm] {\scriptsize$\boldsymbol{s}_9(t+1)$};
\node (s10) at (6,\levelOneSize) [draw, circle, align=center, inner sep=0pt, minimum size=1cm] {\scriptsize$\boldsymbol{s}_{10}(t+1)$};
\node (s11) at (7.5,\levelOneSize) [draw, circle, align=center, inner sep=0pt, minimum size=1cm] {\scriptsize$\boldsymbol{s}_{11}(t+1)$};
\node (s12) at (9,\levelOneSize) [draw, circle, align=center, inner sep=0pt, minimum size=1cm] {\scriptsize$\boldsymbol{s}_{12}(t+1)$};

\draw[->] (s0) to node[sloped, above] {\scriptsize$\Pr[q_1(t+1)=G|q_1(t)=G]\cdot$} node[sloped, below] {\scriptsize$\Pr[q_2(t+1)=G| q_2(t)=G$} (q1);
\path[->] (s0) edge node [sloped, above] {\scriptsize$\Pr[G|G]\cdot \Pr[B|G]$} (q2);
\path[->] (s0) edge node [sloped, above] {\scriptsize$\Pr[B|G]\cdot \Pr[G|G]$} (q3);
\path[->] (s0) edge node [sloped, above] {\scriptsize$\Pr[B|G]\cdot \Pr[B|G]$} (q4);

\path[->] (q1) edge node [sloped, above] {\scriptsize$\Pr[\gamma_{1}(t)=1|\mu(t)=1]$} (s1);
\path[->] (q1) edge node [sloped, rotate=180, above] {\scriptsize$\Pr[1| 2]$} (s2);
\path[->] (q1) edge node [sloped, above]{\scriptsize$\Pr[0|1]$} (s3);
\path[->, bend left=30] (q1) edge node [sloped, above]{\scriptsize$\Pr[0|2]$} (s3);
\path[->, bend left=60] (q1) edge node [sloped, above]{\scriptsize$\Pr[0|\varnothing]$} (s3);

\path[->] (q2) edge node [sloped, above] {\scriptsize$\Pr[\gamma_{1}(t)=1|\mu(t)=1]$} (s4);
\path[->] (q2) edge node [sloped, rotate=180, above] {\scriptsize$\Pr[1| 2]$} (s5);
\path[->] (q2) edge node [sloped, above]{\scriptsize$\Pr[0|1]$} (s6);
\path[->, bend left=30] (q2) edge node [sloped, above]{\scriptsize$\Pr[0|2]$} (s6);
\path[->, bend left=60] (q2) edge node [sloped, above]{\scriptsize$\Pr[0|\varnothing]$} (s6);

\path[->] (q3) edge node [sloped, above] {\scriptsize$\Pr[\gamma_{1}(t)=1|\mu(t)=1]$} (s7);
\path[->] (q3) edge node [sloped, rotate=180, above] {\scriptsize$\Pr[1| 2]$} (s8);
\path[->] (q3) edge node [sloped, above]{\scriptsize$\Pr[0|1]$} (s9);
\path[->, bend left=30] (q3) edge node [sloped, above]{\scriptsize$\Pr[0|2]$} (s9);
\path[->, bend left=60] (q3) edge node [sloped, above]{\scriptsize$\Pr[0|\varnothing]$} (s9);

\path[->] (q4) edge node [sloped, above] {\scriptsize$\Pr[\gamma_{1}(t)=1|\mu(t)=1]$} (s10);
\path[->] (q4) edge node [sloped, rotate=180, above] {\scriptsize$\Pr[1| 2]$} (s11);
\path[->] (q4) edge node [sloped, above]{\scriptsize$\Pr[0|1]$} (s12);
\path[->, bend left=30] (q4) edge node [sloped, above]{\scriptsize$\Pr[0|2]$} (s12);
\path[->, bend left=60] (q4) edge node [sloped, above]{\scriptsize$\Pr[0|\varnothing]$} (s12);

\node (lev0) [] at (-10.5,0) {Level 0};
\draw [dashed,mygray, very thick] (lev0.east) to node [] (xyz) {} (s0.west);
\draw [dashed,mygray, very thick] (s0.east) to node [] (xyz) {} (10,0);

\node (lev1) [] at (-10.5,\levelOneSize) {Level 1};

\draw [dashed, mygray,very thick] (lev1.east) to node [] (xyz) {} (s1.west);
\draw [dashed, mygray,very thick] (s1.east) to node [] (xyz) {} (s2.west);
\draw [dashed, mygray,very thick] (s2.east) to node [] (xyz) {} (s3.west);
\draw [dashed, mygray,very thick] (s3.east) to node [] (xyz) {} (s4.west);
\draw [dashed, mygray,very thick] (s4.east) to node [] (xyz) {} (s5.west);
\draw [dashed, mygray,very thick] (s5.east) to node [] (xyz) {} (s6.west);
\draw [dashed, mygray,very thick] (s6.east) to node [] (xyz) {} (s7.west);
\draw [dashed, mygray,very thick] (s7.east) to node [] (xyz) {} (s8.west);
\draw [dashed, mygray,very thick] (s8.east) to node [] (xyz) {} (s9.west);
\draw [dashed, mygray,very thick] (s9.east) to node [] (xyz) {} (s10.west);
\draw [dashed, mygray,very thick] (s10.east) to node [] (xyz) {} (s11.west);
\draw [dashed, mygray,very thick] (s11.east) to node [] (xyz) {} (s12.west);

\end{tikzpicture}
 
  \caption[GES: Example tree structure with 2 sub-systems and finite horizon
  1]{Example $1$ level tree structure in a GE channel with $2$ sub-systems,
  i.e., $H=1$ and $N=2$. An initial state $\boldsymbol{s}(t)=\left[t\quad t\quad
  a\quad b\quad c\quad d\quad G\quad G \right]^T$ is assumed. $\boldsymbol{q}_
  {1\dots 4}$ resemble the $2^N=4$ possible GE channel states, each labeled with
  the corresponding transition probability, i.e., $\prod_{i=1}^{N}{\Pr[q_i(t+1)
  = G|q_i(t)=G]}$. Each of these intermediate nodes form the root of the tree
  shown in Fig.~(\ref{fig:FHStree}). In these $\Pr[\gamma_{\mu(t)}(t)\mid
  \mu(t)]$ stands for the conditional probability for the possible next network
  state to occur given the scheduling decision. For each $\boldsymbol{q}_{1\dots
  4}$, the first two child nodes correspond to the success state while the third
  child node is the shared failed state.}
	\label{fig:GEStree}
\end{sidewaysfigure}

Furthermore, it should be emphasized, that although the obtained optimal policy
$\pi^*$ consists of $H$ optimal scheduling decisions, only the action at level 0
is applied. This is because the root of the tree changes after each slot,
causing the remaining $H-1$ actions to be not optimal anymore. Hence, steps
(1)-(6) have to be repeated every time slot as well. 

\subsection{Complexity}

The complexity of the GES is governed by both the construction of the tree and
the search for the shortest path. The worst-case scenario in terms of complexity
takes place when all sub-systems in the network are sampled every slot, i.e.,
$D_i=1$. Then every node in the tree will have $2^N(N+1)$ child nodes. The total
amount of nodes $n$ in a perfect $m$-ary tree is computed by:

\begin{equation}
  n = \sum_{i=0}^{h}m^i = \frac{m^{h+1}-1}{m-1}
\end{equation}
Applied to our tree with $m=2^N(N+1)$ and $h=H$: 
\begin{equation}
  n = \frac{(2^N(N+1))^{H+1}-1}{2^N(N+1)-1} 
\end{equation} 

Depending on how often each node is visited in the concrete implementation, the
complexity becomes $c \cdot n$, where $c$ is a positive constant. However, in
simplified Big-O notation, $c$ can be omitted and the complexity is given as
$\mathcal{O}(2^N\cdot N^H)$. This corresponds to a brute-force search for the shortest
path in our constructed tree. The amount of nodes and complexity can be reduced
by increasing the sampling period. As packets are only generated at sampling
events, if a sub-systems succeeds to transmit its plant measurement from that
sampling period, it will not have any new packet to transmit for the rest of
sampling period. Thus, the number of possible actions at these states will not
incorporate said sub-systems. In other words, if $t_{i,g}(t) = t_{i,r}(t)$, then
$i \not \in \mathcal{M}(t)$, thereby, reducing the amount of nodes in the next
level.
