\chapter{Scenario and Problem Statement}

Suppose $N$ independent, linear time-invariant (LTI) control systems sharing a
wirless communication network. Each individual subsytem $i$ consists of a plant
$\plant$, a sensor $\sensor$, and a controller $\controller$ with an estimator
$\estimator$. Each sensor $\sensor$ samples the plant periodically and transmits
the latest state information to $\controller$. On the controller side,
$\estimator$ estimates periodically the current plant state based on the latest
received information. Estimated state is then used by $\controller$ to calculate
the next control input. We assume each controller-plant pair to be co-located
and hence connected through an ideal link while the sensor is operating remotely
over the shared wirless channel as illustrated in Fig. \ref{fig:scenario}.

\begin{figure}[h]
  \centering
  \begin{tikzpicture}[>=latex, scale=0.9]
  \node (plant) at (0.4,6.4) [draw,minimum width=2cm,minimum height=1.5cm, rounded corners=0.1cm, text width=3cm, align=center, fill=mylightestgray] {};
  
  \node (plant) at (0.2,6.2) [draw,minimum width=2cm,minimum height=1.5cm, rounded corners=0.1cm, text width=3cm, align=center, fill=mylightergray] {};
  
  % Plant
  \node (plant) at (0,6) [draw,minimum width=2cm,minimum height=1.5cm, rounded corners=0.1cm, text width=3cm, align=center, fill=mygray] {\large{Plant $\mathcal{P}_i$}};
  
  
  % Sensor 
  \node (sensor) at (12.4,6.4) [draw,minimum width=2cm,minimum height=1.5cm, rounded corners=0.1cm, text width=3cm, align=center, fill=mylightestgray] {};
  
  \node (sensor) at (12.2,6.2) [draw,minimum width=2cm,minimum height=1.5cm, rounded corners=0.1cm, text width=3cm, align=center, fill=mylightergray] {};
  
  \node (sensor) at (12,6) [draw,minimum width=2cm,minimum height=1.5cm, rounded corners=0.1cm, text width=3cm, align=center, fill=lightgray] {\large{Sensor $\mathcal{S}_i$}};
  
  %Estimator
  \node (estimator) at (6.4,2.4) [draw,minimum width=2cm,minimum height=1.5cm, rounded corners=0.1cm, text width=3cm, align=center, fill=mylightestgray] {\large{Estimator $\mathcal{E}_i$}};

  \node (estimator) at (6.2,2.2) [draw,minimum width=2cm,minimum height=1.5cm, rounded corners=0.1cm, text width=3cm, align=center, fill=mylightergray] {\large{Estimator $\mathcal{E}_i$}};

  \node (estimator) at (6,2) [draw,minimum width=2cm,minimum height=1.5cm, rounded corners=0.1cm, text width=3cm, align=center, fill=lightgray] {\large{Estimator $\mathcal{E}_i$}};


  % Controller
  \node (controller) at (0.4,2.4) [draw,minimum width=2cm,minimum height=1.5cm, rounded corners=0.1cm, text width=3cm, align=center, fill=mylightestgray] {};
  
  \node (controller) at (0.2,2.2) [draw,minimum width=2cm,minimum height=1.5cm, rounded corners=0.1cm, text width=3cm, align=center, fill=mylightergray] {};
  
  \node (controller) at (0,2) [draw,minimum width=2cm,minimum height=1.5cm, rounded corners=0.1cm, text width=3cm, align=center, fill=lightgray] {\large{Controller $\mathcal{C}_i$}};
  
  
  % Network Cloud
  \node[cloud, cloud puffs=11, cloud ignores aspect, minimum width=2cm, minimum height=1.5, text width=2.5cm, align=center, draw] (cloud) at (12, 2) {\large {Network \\} \small{with scheduler}};
  
  % C-to-P arrow
  \draw[arrows=-triangle 45,black, ultra thick] (controller.north) -- node[pos=0.5, left] {\large $u_i[k_i]$} (plant.south);
  
  % P-to-S arrow
  \draw[arrows=-triangle 45,black, ultra thick] (plant.east) -- node[pos=0.5, above] {\large $x_i[k_i]$} (sensor.west);
  
  % P-to-Cloud arrow
  \draw[black, dashed, ultra thick] (sensor.south) -- node[] {} (cloud.north);
  
  % Cloud-to-E arrow
  \draw[arrows=-triangle 45,black, dashed, ultra thick] (cloud.west) -- node[pos=0.5, above] {} (estimator.east);

  % E-to-C arrow
  \draw[arrows=-triangle 45,black, dashed, ultra thick] (estimator.west) -- node[pos=0.5, above] {\large $\hat{x}_i[k_i]$} (controller.east);
  
  \end{tikzpicture}

  \caption[Scheme of $N$ subsytems sharing a wirelss communication medium]{Considered scenario with $N$ heterogenous LTI networked control systems. Solid lines indicate ideal controller-to-plant and plant-to-sensor links. Sensor-to-contoller link is closed over a shared wireless channel. Medium access is granted centrally by a scheduler. Note that $k_i$ refers to the sampling period a sub-system $i$ is in.}
  \label{fig:scenario}
\end{figure}


\section{Network Model}
Time is divided into slots which is also the smallest time unit in our scenario.
We use $t \in \mathbb{N}$ All transmissions are managed by a centralized entity
called \textit{scheduler}.

\section{Age-of-Information Model}

\section{Control Model}

\section{Scheduling Problem Formulation} \label{sec:problem}
